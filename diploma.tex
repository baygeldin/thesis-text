\documentclass[14pt]{matmex-diploma-custom}

\begin{document}
\filltitle{ru}{
  chair              = {Программная инженерия},
  title              = {Система сбора и анализа показателей жизнедеятельности на основе данных с мобильных устройств},
  type               = {bachelor},
  position           = {студента},
  group              = 471,
  author             = {Байгельдин Александр Юрьевич},
  supervisorPosition = {к.\,ф.-м.\,н., доцент},
  supervisor         = {Романовский К.\,Ю.},
  reviewerPosition   = {Зам. ген. директора ООО ``ПитерСофтвареХаус''},
  reviewer           = {Хитров Д.\,В.},
  year               = {2018}
}
\filltitle{en}{
  chair              = {Software Engineering},
  title              = {System for collection and analysis of vital factors based on data from mobile devices},
  author             = {Aleksandr Baigeldin},
  supervisorPosition = {assistant professor},
  supervisor         = {Konstantin Romanovsky},
  reviewerPosition   = {Deputy director general of ``PiterSoftwareHouse'' Ltd.},
  reviewer           = {Denis Khitrov},
  university         = {Saint Petersburg State University},
}
\maketitle
\tableofcontents

\section*{Введение}
  Стресс --- ЭТО ПЛОХО. А он так-то повсюду. $C$, $C++$, $Java$ --- это все стресс.

  Но еще стоит заметить, что сейчас популярны медицинские сенсоры.

  А еще машинное обучение все больше применяется в медицине.

  Таким образом, давайте делать систему с применением машинного обучения, которая будет определять стресс на основе данных с медицинских сенсоров.
	
\section{Постановка задачи}


\begin{itemize}
  \item Lorem ipsum.
    \item Dolor sit amet.
  \item Consectetur adipiscing elit.
  \item Sed do eiusmod tempor incididunt ut labore et dolore magna aliqua.
\end{itemize}

\section*{Заключение}
  Lorem ipsum \cite{saturday_is_monday}.

\setmonofont[Mapping=tex-text]{CMU Typewriter Text}
\bibliographystyle{assets/ugost2008ls}
\bibliography{diploma}
\end{document}

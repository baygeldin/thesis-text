\documentclass[14pt]{matmex-diploma-custom}

\begin{document}
\filltitle{ru}{
  chair = {Программная инженерия},
  title = {Система сбора и анализа показателей жизнедеятельности на основе данных с мобильных устройств},
  type = {bachelor},
  position = {студента},
  group = 471,
  author = {Байгельдин Александр Юрьевич},
  supervisorPosition = {к.\,ф.-м.\,н., доцент},
  supervisor = {Романовский К.\,Ю.},
  reviewerPosition = {Зам. ген. директора ООО ``ПитерСофтвареХаус''},
  reviewer = {Хитров Д.\,В.},
  year = {2018}
}
\filltitle{en}{
  chair = {Software Engineering},
  title = {System for collection and analysis of vital factors based on data from mobile devices},
  author = {Aleksandr Baigeldin},
  supervisorPosition = {assistant professor},
  supervisor = {Konstantin Romanovsky},
  reviewerPosition = {Deputy director general of ``PiterSoftwareHouse'' Ltd.},
  reviewer = {Denis Khitrov},
  university = {Saint Petersburg State University},
}
\maketitle
\tableofcontents

\section*{Введение}



Стресс это... Стресс иногда хорошо, но обычно плохо. Стресс вреден, потому
что...=цитата Стресса стало слишком много. Зачастую мы просто не знаем, что у
нас стресс. Нужно научиться его предупреждать. Значит нужно научиться его
определять -- в реальном времени.

Тут есть два способа --- инвазивный и мониторить HPA, и неинвазивный чтобы
мониторить симпатическую нервную систему. =цитата? Инвазивный для мобильности не
подходит, также дорого. Берем симпатическую систему. Симпатическая система
влияет сразу на многое -- бладпрешшуре, выделение пота, пульс.=цитата? Для всего
этого есть популярные сейчас носимые сенсоры. Но это все неудобно, кое-что
ненадежно, кое-что дорого. Пульсометры збс. Самое збс -- это вариабиельность
(что это...). Почему вариабиельность важна =цитата.

Однако, вариабельность как и другие показатели не стабильны и меняются при
небольших стрессорах, в частности активности (когда мы ходим, симпатическая
ветвь работает больше =цитата). Таким образом надо учитывать активность. Но по
сырым данным человек не многое поймет, поэтому применим популярное в медицине
=цитата машинное обучение!

Таким образом, давайте делать систему с применением машинного обучения, которая
будет определять стресс на основе данных с медицинских сенсоров в реальном
времени.
	
\section{Постановка задачи}
Целью данной работы является создание прототипа системы для определения
человеческого стресса в реальном времени на основе данных полученных с мобильных
устройств.

Для достижения этой цели были сформулированы следующие задачи:

\begin{itemize}
\item Ознакомиться с природой человеческого стресса и изучить публикации на тему
  предсказания стресса на основе медицинских данных.
\item Спроектировать систему для определения стресса в реальном времени на
  основе данных с пульсометра и акселерометра.
\item Написать мобильное приложение для сбора данных и выделения из них
  признаков, полезных для анализа стресса.
\item Выбрать способ сбора данных, натренировать модель на собранных данных,
  оценить ее эффективность и интегрировать ее в приложение, чтобы достичь
  анализа стресса в реальном времени.
\end{itemize}

\section*{Заключение}
Lorem ipsum \cite{saturday_is_monday}.

\setmonofont[Mapping=tex-text]{CMU Typewriter Text}
\bibliographystyle{assets/ugost2008ls}
\bibliography{diploma}
\end{document}

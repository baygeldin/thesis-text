\documentclass[14pt]{matmex-diploma-custom}

\begin{document}
\filltitle{ru}{
  chair = {Программная инженерия},
  title = {Система сбора и анализа показателей жизнедеятельности на основе данных с мобильных устройств},
  type = {bachelor},
  position = {студента},
  group = 471,
  author = {Байгельдин Александр Юрьевич},
  supervisorPosition = {к.\,ф.-м.\,н., доцент},
  supervisor = {Романовский К.\,Ю.},
  reviewerPosition = {Зам. ген. директора ООО ``ПитерСофтвареХаус''},
  reviewer = {Хитров Д.\,В.},
  year = {2018}
}
\filltitle{en}{
  chair = {Software Engineering},
  title = {System for collection and analysis of vital factors based on data from mobile devices},
  author = {Aleksandr Baigeldin},
  supervisorPosition = {assistant professor},
  supervisor = {Konstantin Romanovsky},
  reviewerPosition = {Deputy director general of ``PiterSoftwareHouse'' Ltd.},
  reviewer = {Denis Khitrov},
  university = {Saint Petersburg State University},
}
\maketitle
\tableofcontents

\section*{Введение}
Стресс --- это совокупность неспецифических (т.е. независимых от типа стрессора)
адаптационных реакций организма на воздействие различных неблагоприятных
факторов (физических или психологических), нарушающих его гомеостаз (стабильное,
равновесное состояние) \cite{book:stress_of_life}. В современной медицине
принято разделять понятие положительного стресса (эустресса) и отрицательного
стресса (дистресса) \cite{article:eustress_distress}. В результате
положительного стресса повышается функциональный резерв организма, происходит
его адаптация к стрессовому фактору и ликвидация самого стресса. Однако, когда
организм постоянно подвергается стрессу или же стрессор слишком сильный,
защитные силы организма истощаются и он становится не в состоянии самостоятельно
справиться со стрессом. От такого стресса страдает иммунная система, он
подрывает здоровье человека и может привести к тяжелым заболеваниям, таким как
депрессивное расстройство, диабет и даже рак \cite{article:stress_and_illness}.
В связи с этим, получили широкое развитие различные методы управления стрессом,
которые помогают предупреждать его отрицательное воздействие. Однако, зачастую
человек не замечает или не осознает того, что подвергается воздействию стресса.
Поэтому перспективной областью исследования является автоматическое отслеживание
стресса в реальном времени.

Поскольку основное воздействие стресс оказывает на нервную и эндокринную системы
организма, то для его определения логичным является поиск соответствующих
паттернов в работе этих систем. Например, анализ крови может выявить повышенное
содержание кортизола (глюкортикоидного ``гормона стресса'') в крови. Однако,
инвазивные методы не подходят для постоянного отслеживания стресса. В связи с
этим, особый интерес вызывает реакция нервной системы организма на стресс, а
если точнее, то реакция симпатического отдела автономной нервной системы,
который отвечает за мобилизацию сил организма в экстренных ситуациях.
Симпатическая нервная система оказывает влияние на частоту сердцебиения и
дыхания, кровяное давление, электрическую активность кожи и другие показатели.
Поэтому диапазон медицинских сенсоров, с помощью которых можно в той или иной
мере определять стресс, довольно обширен: пульсометры, тонометры, GSR сенсоры, и
т.д. Тем не менее, наиболее перспективным типом сенсоров для задачи отслеживания
стресса в реальном времени кажутся именно пульсометры, т.к. несмотря на
небольшую цену, они обладают необходимой мобильностью и предоставляют
возможность высчитывать один из самых важных показателей активности
симпатической нервной системы --- вариабельность сердечного ритма
\cite{article:hrv_stress}.

Однако, симпатическая нервная система реагирует даже на небольшие стрессоры,
которые нет смысла учитывать в статистике, но которые при этом оказывают влияние
на вариабельность сердечного ритма. Например, даже при медленной ходьбе
вариабельность сердечного ритма отличается от сидячего положения
\cite{article:hrv_reliability}, хотя нельзя назвать ходьбу стрессом в
отрицательном смысле. Поэтому учет физической активности (например, на основе
данных с акселерометра) является хорошим способом отфильтровать ложные
срабатывания отслеживающей стресс системы. Для комбинации показателей физической
активности и показателей активности симпатической нервной системы при
определении стресса можно применить популярное на сегодняшний день в медицине
машинное обучение.

Таким образом, для задачи автоматического отслеживания стресса в реальном
времени требуется система, которая бы определяла стресс на основе данных с
пульсометра и акселерометра.
	
\section{Постановка задачи}
Целью данной работы является создание прототипа системы для определения
человеческого стресса в реальном времени на основе данных полученных с мобильных
устройств.

Для достижения этой цели были поставлены следующие задачи:

\begin{itemize}
\item Ознакомиться с природой человеческого стресса и изучить публикации на тему
  предсказания стресса на основе медицинских данных.
\item Спроектировать систему для определения стресса в реальном времени на
  основе данных с пульсометра и акселерометра.
\item Написать мобильное приложение для сбора данных и выделения из них
  признаков, полезных для определения стресса.
\item Выбрать способ сбора данных, натренировать модель на собранных данных,
  оценить ее эффективность и интегрировать ее в приложение, чтобы достичь
  определения стресса в реальном времени.
\end{itemize}

\setmonofont[Mapping=tex-text]{CMU Typewriter Text}
\bibliographystyle{assets/ugost2008ls} \bibliography{diploma}
\end{document}
